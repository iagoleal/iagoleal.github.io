\documentclass{standalone}
% For matrix loop see: https://tex.stackexchange.com/questions/47595/nested-foreach-inside-a-tikz-matrix-for-both-rows-and-columns
% For graph from array see: https://tex.stackexchange.com/questions/202151/how-to-plot-a-graph-from-its-adjacency-matrix-and-coordinates-of-vertices#203070

\def\pgfsysdriver{pgfsys-dvisvgm.def}
\usepackage{tikz,pgffor,ifthen}
\usepackage{etoolbox}

\usetikzlibrary{positioning, arrows, arrows.meta, shapes.geometric}
\usetikzlibrary {decorations.pathmorphing, chains, calc, matrix}

\begin{document}

\colorlet{smoothgray}{gray!80}
\colorlet{smoothorange}{orange!70}

% Reflexive-transitive closure
% 0 => no edge
% 1 => original relation
% 2 => only on closure
\def\adjacency{{1,1,2,0}%
              ,{1,2,1,0}%
              ,{0,0,2,0}%
              ,{1,2,2,1}}

\makeatletter
\tikzset{
    to path={
        \pgfextra{
            \tikz@scan@one@point\relax(\tikztostart)
            \pgfmathsetmacro\start@x{\pgf@x}
            \pgfmathsetmacro\start@y{\pgf@y}
            \tikz@scan@one@point\relax(\tikztotarget)
            \pgfmathsetmacro\target@x{\pgf@x}
            \pgfmathsetmacro\target@y{\pgf@y}
            \pgfmathsetmacro\middle@x{(\start@x+\target@x)/2}
            \pgfmathsetmacro\middle@ya{(4*\start@y+\target@y)/5}
            \pgfmathsetmacro\middle@yb{(\start@y+4*\target@y)/5}
        }
        ..controls(\middle@x pt,\middle@ya pt)and(\middle@x pt,\middle@yb pt)..(\tikztotarget)
    }
}

\begin{tikzpicture}[
    state/.style = {circle, radius = 0.9cm, minimum width = 0.7cm, fill = gray!20, draw = gray!40, very thick },
    rel/.style   = {shorten >= 1mm, shorten <= 1mm, draw = blue!70, thick},
    baseline=(current bounding box.south),
  ]

  % Relations
  \begin{scope}[ start chain        = source going below
               , local bounding box = relScope
               , node distance      = 2mm
               , baseline           = {(0,0)}
               ]
    \foreach[count=\n] \row in \adjacency {
      \node[state, on chain = source] (s\n) {$\n$};
      \node[state] (t\n) [right = 20mm of s\n.east] {$\n$};
    }

    \foreach[count=\s] \row in \adjacency {
      \foreach[count=\t] \cell in \row {
          \ifnum\cell=1
            \draw[rel] (s\s.east) -- (t\t.west);
          \fi
      }
    }
  \end{scope}

  % Complete relation
  \begin{scope}[ start chain        = source going below
               , local bounding box = relCloScope,
               , node distance      = 2mm
               , xshift = 7cm,
               ]
    \foreach[count=\n] \row in \adjacency {
      \node[state, on chain = source] (s\n) {$\n$};
      \node[state] (t\n) [right = 20mm of s\n.east] {$\n$};
    }

    \foreach[count=\s] \row in \adjacency {
      \foreach[count=\t] \cell in \row {
          \ifnum\cell=1
            \draw[rel] (s\s.east) -- (t\t.west);
          \else
            \ifnum\cell=2
              \draw[rel, draw = blue!30, semithick] (s\s.east) -- (t\t.west);
            \fi
          \fi
      }
    }
  \end{scope}

  % \draw ($(relScope.east)$)
  \draw[->, -Latex, thick, shorten >= 4mm, shorten <= 4mm] (relScope.east) -- node[above] {closure} (relCloScope.west);
\end{tikzpicture}

\end{document}
