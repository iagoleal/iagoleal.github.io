\documentclass{standalone}
% For matrix loop see: https://tex.stackexchange.com/questions/47595/nested-foreach-inside-a-tikz-matrix-for-both-rows-and-columns
% For graph from array see: https://tex.stackexchange.com/questions/202151/how-to-plot-a-graph-from-its-adjacency-matrix-and-coordinates-of-vertices#203070

\def\pgfsysdriver{pgfsys-dvisvgm.def}
\usepackage{lmodern,tikz,pgffor,ifthen}
\usepackage{etoolbox}

\usetikzlibrary{positioning, arrows, arrows.meta, shapes.geometric}
\usetikzlibrary {decorations.pathmorphing, chains, calc, matrix, quotes}

\begin{document}

\colorlet{smoothgray}{gray!80}
\colorlet{bluestrong}{blue!70}
\colorlet{blueweak}{blue!40}

% Reflexive-transitive closure
% 0 => no edge
% 1 => original relation
% 2 => only on closure
\def\adjacency{{1,1,2,0}%
              ,{1,2,1,0}%
              ,{0,0,2,0}%
              ,{1,2,2,1}}

\makeatletter
\tikzset{
    to path={
        \pgfextra{
            \tikz@scan@one@point\relax(\tikztostart)
            \pgfmathsetmacro\start@x{\pgf@x}
            \pgfmathsetmacro\start@y{\pgf@y}
            \tikz@scan@one@point\relax(\tikztotarget)
            \pgfmathsetmacro\target@x{\pgf@x}
            \pgfmathsetmacro\target@y{\pgf@y}
            \pgfmathsetmacro\middle@x{(\start@x+\target@x)/2}
            \pgfmathsetmacro\middle@ya{(4*\start@y+\target@y)/5}
            \pgfmathsetmacro\middle@yb{(\start@y+4*\target@y)/5}
        }
        ..controls(\middle@x pt,\middle@ya pt)and(\middle@x pt,\middle@yb pt)..(\tikztotarget)
    }
}

\begin{tikzpicture}[
      state/.style = {circle, minimum width = 0.7cm, fill = gray!20, draw = gray!40, very thick }
    , rel/.style   = {shorten >= 1mm, shorten <= 1mm, draw = bluestrong, thick}
    , loop arc/.style={in=240,out=200,loop,min distance=4mm}
    , action/.style   = {
        bluestrong,
        ->,
        >={Kite[length=4pt,width=2.5pt,inset=1.5pt]},
        thick,
    }
    , baseline=(current bounding box.south)
    ]

  % Relations
  \begin{scope}[ start chain        = source going below
               , local bounding box = relScope
               , node distance      = 2mm
               , baseline           = {(0,0)}
               ]
    \foreach[count=\n] \row in \adjacency {
      \node[state, on chain = source] (s\n) {$\n$};
      \node[state] (t\n) [right = 20mm of s\n.east] {$\n$};
    }

    \foreach[count=\s] \row in \adjacency {
      \foreach[count=\t] \cell in \row {
          \ifnum\cell=1
            \draw[rel] (s\s.east) -- (t\t.west);
          \fi
      }
    }
  \end{scope}

  % Complete relation
  \begin{scope}[ start chain        = source going below
               , local bounding box = relCloScope,
               , node distance      = 2mm
               , yshift = -5cm,
               ]
    \foreach[count=\n] \row in \adjacency {
      \node[state, on chain = source] (s\n) {$\n$};
      \node[state] (t\n) [right = 20mm of s\n.east] {$\n$};
    }

    \foreach[count=\s] \row in \adjacency {
      \foreach[count=\t] \cell in \row {
          \ifnum\cell=1
            \draw[rel] (s\s.east) -- (t\t.west);
          \else
            \ifnum\cell=2
              \draw[rel, draw = blueweak, semithick] (s\s.east) -- (t\t.west);
            \fi
          \fi
      }
    }
  \end{scope}

  % Adjacency matrix
  \begin{scope}[shift={($(relScope.east)+(3cm,0)$)},
                local bounding box = matScope]
    % First of all, we write a macro to build the matrix content.
    % This is necessary because there is no way to put a \foreach
    % inside a \matrix command.
    \let\mymatrixcontent\empty
    \foreach[count=\s] \row in \adjacency {
      \foreach[count=\t] \cell in \row {
        \xappto\mymatrixcontent{\ifnum\cell=1 1 \else 0 \fi \expandonce{\&}}
      }%
      \gappto\mymatrixcontent{\\}
    }%%

    \matrix[matrix of math nodes, ampersand replacement=\&,
            left delimiter = {[}, right delimiter = {]},
            column sep=1ex,
            ] (m)
      { \mymatrixcontent };
  \end{scope}

  \begin{scope}[shift={($(relCloScope.east)+(3cm,0)$)},
                local bounding box = matCloScope]
    \let\mymatrixcontent\empty
    \foreach[count=\s] \row in \adjacency {
      \foreach[count=\t] \cell in \row {
        \xappto\mymatrixcontent{\ifnum\cell>0 \ifnum\cell=2 |[blueweak]| \fi 1 \else 0 \fi \expandonce{\&}}
      }%
      \gappto\mymatrixcontent{\\}
    }%%

    \matrix[matrix of math nodes, ampersand replacement=\&,
            left delimiter = {[}, right delimiter = {]},
            column sep=1ex,
            ] (m)
      { \mymatrixcontent };
  \end{scope}

  % Graph
  \begin{scope}[ shift={($(matScope.east)+(4cm,0)$)},
               , local bounding box = graphScope
               ]

    % Nodes
    \foreach [count=\s] \where in {(0, 0), (-2, 1), (-2, -1), (0, -1.5)} {
      \node[state] (g\s) at \where {$\s$};
    }

    % Edges from adjacency matrix
    \foreach[count=\s] \row in \adjacency {
      \foreach[count=\t] \cell in \row {
        \ifnum\cell=1
          \ifnum\s=\t
            \path[action] (g\s) edge [loop arc]  (g\s);
          \else
            \path[action] (g\s) edge [bend right=30]  (g\t);
          \fi
        \fi
      }
    }
  \end{scope}

  \begin{scope}[ shift={($(matCloScope.east)+(4cm,0)$)},
               , local bounding box = graphCloScope
               ]

    % Nodes
    \foreach [count=\s] \where in {(0, 0), (-2, 1), (-2, -1), (0, -1.5)} {
      \node[state] (g\s) at \where {$\s$};
    }

    % Edges from adjacency matrix
    \foreach[count=\s] \row in \adjacency {
      \foreach[count=\t] \cell in \row {
        \ifnum\cell>0
          \pgfmathsetmacro\ncol{iseven(\cell) ? "blueweak": "bluestrong"}
          \ifnum\s=\t
            \path[action] (g\s) edge [loop arc]  (g\s);
          \else
            \path[action, \ncol] (g\s) edge [bend left=10]  (g\t);
          \fi
        \fi
      }
    }
  \end{scope}

  \begin{scope}[arrow/.style = {-Latex, thick, shorten >= 2mm, shorten <= 2mm}
               , iso/.style  = {arrow, Latex-Latex}
               ]
    % Closure arrows
    \path[arrow] let \p1 = (relScope.south), \p2 = (matScope.south), \p3 = (graphScope.south)
                 in (\x1, \y1) edge  ++(0cm, -1.5cm)
                    (\x2, \y1) edge  ++(0cm, -1.5cm)
                    (\x3, \y1) edge  ++(0cm, -1.5cm);

    % Adjacency arrows
    \path[iso] (relScope.east)    edge +(1.5cm, 0cm)
               (matScope.east)    edge +(1.5cm, 0cm)
               (relCloScope.east) edge +(1.5cm, 0cm)
               (matCloScope.east) edge +(1.5cm, 0cm);
  \end{scope}
\end{tikzpicture}

\end{document}
