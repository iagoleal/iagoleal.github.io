\documentclass{standalone}

\def\pgfsysdriver{pgfsys-dvisvgm.def}
\usepackage{tikz}

\usetikzlibrary{animations,positioning, arrows, arrows.meta, backgrounds, fit}

\begin{document}

\colorlet{smoothgray}{gray!80}

\begin{tikzpicture}[
  node distance = 3cm,
  state/.style    = { circle, minimum width = 0.7cm, fill = gray!20, draw = gray!40, very thick },
  terminal/.style = {rectangle, fill = black, scale = 1},
  action/.style   = {
    opacity = 0.8,
    thin,
    smoothgray,
    ->,
    >={Kite[length=4pt,width=2.5pt,inset=1.5pt]},
  },
  bg/.style = {
    rectangle,
    rounded corners,
    inner sep      = 0.35cm,
    minimum height = #1*1.5cm,
    minimum width  = 1.5cm,
  },
  animate/glow/.style n args ={3}{
    myself:opacity = {base = "#1", 0s="current value", 0.3s="#2", freeze, begin on={mouse over, of next = #3}},
    myself:opacity = {id = 2, 0s="current value", 0.3s="#1", freeze, begin on= {mouse out, of next = #3} },
  },
]

  % Useful constants
  % How many states per stage
  \def\states{{1,3,2,3}}
  % Total number of stages
  \pgfmathsetmacro{\nStages}{dim(\states)}
  % Maximum number of states among stages
  \expandafter\pgfmathmax\states{}\edef\maxStates{\pgfmathresult}

  \pgfmathsetseed{5671243}

  \foreach \t in {\nStages,...,1} {
    \pgfmathsetmacro{\nStates}{\states[\t-1]}

    \foreach \s in {\nStates,...,1} {
      % Grid of states
      \node [state, outer sep = 2] (q\t\s) at ({3cm*(\t-1)}, {1.5cm*(\s - (\nStates+1)/2)}) {};

      % Arraws going out of each state
      \ifnum \t<\nStages
        \pgfmathtruncatemacro{\next}{\t + 1}
        \pgfmathsetmacro{\nStatesNext}{\states[\t]}

        \foreach \ny in {1,...,\nStatesNext}
          \draw [action] (q\t\s) -- (q\next\ny);
      \fi
    }

    \scoped [on background layer]
      \node (bg\t) [animate={glow={0.2}{0.4}{b\t}}, bg=\maxStates, fill = orange, label=above:{$t = \t$}]
                at ({3*(\t-1)}, 0) {};

    \node (b\t) [bg=\maxStates, fill = white, opacity = 0.0,
                  fit=(bg\t)] at (bg\t) {};
  }
\end{tikzpicture}

\end{document}

